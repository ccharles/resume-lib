%%%%%%%%%%%%%%%%%%%%%%%%%%%%%%%%%%%%%%%%%%%%%%%%%%%%%%%%%%%%%%%%%%%%%%%%%%%%%
%
%    Filename: sample-resume.tex
%      Author: Chris Charles
%
% ---------------------------------------------------------------------------
%
%    A sample resume using my resume library.
%
%%%%%%%%%%%%%%%%%%%%%%%%%%%%%%%%%%%%%%%%%%%%%%%%%%%%%%%%%%%%%%%%%%%%%%%%%%%%%


%% This is required. Trust me, leave it in!
\starttext



%% Your contact information. This sets up the contact information block at
%% the top of the page, as well as the footer text.
\ContactInfo{Robert Terwilliger} %% Name
    {Springfield Penitentiary, Springfield, USA} %% Address
    {555-555-5555} %% Phone number. I like to use \bf{} to make it stand out.
    {robert@diebartdie.com} %% Email address. This will be hyperlinked.



%% Each main part of your resume is a section, beginning
%% with \StartSetion{Section Name} and ending at the matching \StopSection.
\StartSection{Highlights}


%% ConTeXt supports many of the constructs that you may have used in
%% LaTeX. Here we're using an unordered list.
\startitemize

\item Excellent narrating and singing voice

\item Dedicated and hard-working

\item Great hair

\stopitemize  %% Many opening tags have matching closing tags

\StopSection %% The Highlights section



\StartSection{Selected professional experience}


%% We normally want more than just an unordered list for item details. Use
%% the \StartOrganization{Organization Name}{Organization location}
%% and \StopOrganization tags to build a detailed sub-section.
\StartOrganization{Municipal government}{Salsiccia, Italy}

%% The \Position{Name}{Start}{End} tag lists a position within an
%% organization. Multiple \Position tags can exist within an Organization,
%% with or without a Responsibilities section.
\Position{Mayor}{2002}{2005}


%% Use \StartResponsibilities and \StopResponsibilities to describe the
%% position. You can use a list instead of paragraphs as appropriate.
\StartResponsibilities

Made wine, wrote budgets, corrected enunciation, organized parades and
feasts, acted in {\it Pagliacci}. Lived a relatively quiet life free of
attempted murder.

\StopResponsibilities


\StopOrganization



\StartOrganization{The Krusty the Klown Show}{Springfield}

\Position{Sideshow Bob}{1987}{1990}

\StartResponsibilities

Acted as Krusty's punching bag, cannon ball, pie target and many other humiliating activites over a period of several years. Helped to raise the ratings of the Krusty the Klown show by framing Krusty for armed robbery.

\StopResponsibilities


%% I've always disliked putting references apart from the position they're
%% acting as reference for. The \Reference{Name}{Title}{Phone}{Email} tag
%% will add an optional reference to this Organization / Position.
\Reference{Herschel Krustofski}{Klown}
    {555-555-5555}{krusty@wahahahahaha.com}

\StopOrganization



\StopSection  %% Professional experience


%% I usually model my Education section the same way as my Experience
%% section. The Organization and Position sections translate quite well.
\StartSection{Education}

\StartOrganization{Yale University}{New Haven, Connecticut}


%% The ending date can be left off if desired
\Position{B.A. in Philosophy}{June, 1975}{}

\StopOrganization

\StopSection


%% This is also required
\stoptext
